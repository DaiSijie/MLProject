%%%%%%%%%%%%%%%%%%%%%%%%%%%%%%%%%%%%%%%%%
% Short Sectioned Assignment
% LaTeX Template
% Version 1.0 (5/5/12)
%
% This template has been downloaded from:
% http://www.LaTeXTemplates.com
%
% Original author:
% Frits Wenneker (http://www.howtotex.com)
%
% License:
% CC BY-NC-SA 3.0 (http://creativecommons.org/licenses/by-nc-sa/3.0/)
%
%%%%%%%%%%%%%%%%%%%%%%%%%%%%%%%%%%%%%%%%%

%----------------------------------------------------------------------------------------
%	PACKAGES AND OTHER DOCUMENT CONFIGURATIONS
%----------------------------------------------------------------------------------------

\documentclass[paper=A4, fontsize=11pt]{scrartcl} % A4 paper and 11pt font size

\usepackage[T1]{fontenc} % Use 8-bit encoding that has 256 glyphs
\usepackage[english]{babel} % English language/hyphenation
\usepackage{amsmath,amsfonts,amsthm} % Math packages
\usepackage{tikz} % to graph
\tikzset{
  treenode/.style  	= 	{shape=rectangle, rounded corners,
                     	draw, align=center,
                     	top color=white, bottom color=blue!20},
  predicat/.style  	= 	{treenode, bottom color=red!30},
  conclusion/.style	= 	{treenode},
}
\usepackage{sectsty} % Allows customizing section commands
\usepackage{url}

\allsectionsfont{\centering \normalfont\scshape} % Make all sections centered, the default font and small caps

\DeclareMathOperator*{\argmin}{argmin}

\usepackage{graphicx}
\usepackage{fancyhdr} % Custom headers and footers
\pagestyle{fancyplain} % Makes all pages in the document conform to the custom headers and footers
\fancyhead{} % No page header - if you want one, create it in the same way as the footers below
\fancyfoot[L]{} % Empty left footer
\fancyfoot[C]{} % Empty center footer
\fancyfoot[R]{\thepage} % Page numbering for right footer
\renewcommand{\headrulewidth}{0pt} % Remove header underlines
\renewcommand{\footrulewidth}{0pt} % Remove footer underlines
\setlength{\headheight}{13.6pt} % Customize the height of the header

%\numberwithin{equation}{section} % Number equations within sections (i.e. 1.1, 1.2, 2.1, 2.2 instead of 1, 2, 3, 4)
%\numberwithin{figure}{section} % Number figures within sections (i.e. 1.1, 1.2, 2.1, 2.2 instead of 1, 2, 3, 4)
%\numberwithin{table}{section} % Number tables within sections (i.e. 1.1, 1.2, 2.1, 2.2 instead of 1, 2, 3, 4)

\setlength\parindent{0pt} % Removes all indentation from paragraphs - comment this line for an assignment with lots of text


%----------------------------------------------------------------------------------------
%	TITLE SECTION
%----------------------------------------------------------------------------------------

\newcommand{\horrule}[1]{\rule{\linewidth}{#1}} % Create horizontal rule command with 1 argument of height

\title{	
\normalfont \normalsize 
\textsc{Johns Hopkins University - Introduction to Machine Learning} \\ [25pt] % Your university, school and/or department name(s)
\horrule{0.5pt} \\[0.4cm] % Thin top horizontal rule
\huge Programming Notes \\ % The assignment title
\horrule{2pt} \\[0.5cm] % Thick bottom horizontal rule
}

\author{Charlie Wang, Gilbert Maystre} % Your name

\date{\normalsize\today} % Today's date or a custom date

\begin{document}

\maketitle % Print the title
\newpage

\section{From paper notation to Java data structures}

\subsection{Forward pass}

\begin{center}
\begin{tabular}{|l|l|} 
	\hline
 		Paper expression & Java\\ 
 	\hline
 	\hline
 		$net_{in_j}$ & \verb!ForwardPassCache.getInputGateInput(j) + forwardCache.getInputGatePeephole(j,0)! \\
 	\hline
  		$y^{in_j}$ & \verb!ForwardPassCache.getInputGateOutput(j)! \\ 
 	\hline
   		$net_{\varphi j}$ & \verb!ForwardPassCache.getForgetGateInput(j) + forwardCache.getForgetGatePeephole(j,0)! \\
 	\hline
   		$y^{\varphi j}$ & \verb!ForwardPassCache.getForgetGateOutput(j)?! \\ 
 	\hline
   		$net_{c_j^v}$ & \verb!ForwardPassCache.?! \\ 
 	\hline
   		$s_{c_j^v}$ & \verb!ForwardPassCache.getCellState(j)! \\ 
 	\hline
   		$net_{out_j}$ & \verb!ForwardPassCache.getOutputGateInput(j) + forwardCache.getOutputGatePeephole(j,0)! \\
 	\hline
   		$y^{out_j}$ & \verb!ForwardPassCache.getOutputGateOutput(j)! \\ 
 	\hline
   		$y^{c_j^v}$ & \verb!ForwardPassCache.getMemoryBlockOutput(j)! \\ 
 	\hline
   		$net_k$ & \verb!ForwardPassCache.getOutputNodeInput(k)! \\ 
 	\hline
   		$y^k$ & \verb!ForwardPassCache.getOutputNodeOutput(k)! \\ 
 	\hline
\end{tabular}
\end{center}

\subsection{Derivative computation}

\begin{center}
\begin{tabular}{|l|l|} 
	\hline
 	Paper expression & Java\\ 
 	\hline
	\hline
 		$dS_{cm}^{jv}$ & \verb!DerivativeCache.getCellDerivative(j, m)! \\ 
	\hline
 		$dS_{in,m}^{jv}$ & \verb!DerivativeCache.getInputGateDerivativeA(j, m)! \\ 
	\hline
  		$dS_{in,c_{j}^{v'}}^{jv}$ & \verb!DerivativeCache.getInputGateDerivativeB(j, vprime)! \\ 
	\hline
  		$dS_{\varphi m}^{jv}$ & \verb!DerivativeCache.getForgetGateDerivativeA(j, m)! \\ 
	\hline
  		$dS_{\varphi,c_{j}^{v'}}^{jv}$ & \verb!DerivativeCache.getForgetGateDerivativeB(j, vprime)! \\ 
	\hline
\end{tabular}
\end{center}

\subsection{Backward pass}
\begin{center}
\begin{tabular}{|l|l|} 
	\hline
 	Paper expression & Java\\ 
 	\hline
	\hline
 		$\Delta w_{km}$ & \verb!BackwardPassCache.getOutputUnit(k,m)! \\ 
 	\hline
 		$\Delta w_{out,m}$ & \verb!BackwardPassCache.getOutputGate(j, m)! \\ 
 	\hline
 		$\Delta w_{out, c_j^v}$ & \verb!BackwardPassCache.getOutputGateC(j)! \\ 
 	\hline
 		$\Delta w_{in,m}$ & \verb!BackwardPassCache.getInputGate(j, m)! \\
 	\hline
 		$\Delta w_{in, c_j^{v'}}$ & \verb!BackwardPassCache.getInputGateC(j, vprime)! \\
 	\hline
 		$\Delta w_{\varphi m}$ & \verb!BackwardPassCache.getForgetGate(j, m)! \\
 	\hline
 		$\Delta w_{\varphi c_j^{v'}}$ & \verb!BackwardPassCache.getForgetGateC(j, vprime)! \\
 	\hline
 		$\Delta w_{c_j^v m}$ & \verb!BackwardPassCache.getCell(j, m)! \\ 
	\hline
\end{tabular}
\end{center}





\newpage


\section{Results}

\newpage


\section{Discussion of the results}

\newpage


\section{Conclusion}


\end{document}